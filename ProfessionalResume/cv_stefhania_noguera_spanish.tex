\noindent
\begin{tabular*}{\textwidth}{l@{\extracolsep{\fill}}l}
\textbf{\Large \href{https://www.linkedin.com/in/stefhania-noguera/}{Stefhania Noguera Romero}} & \href{mailto:stefhanianogueraromero@gmail.com}{Correo: stefhanianogueraromero@gmail.com}\\
Ingeniera de Sistemas y Ciencias de la Computación & \href{https://www.linkedin.com/in/stefhania-noguera/}{LinkedIn: Stefhania Noguera} \\
Ingeniera Full-Stack & Teléfono: (+57) 322 6352232
\end{tabular*}

\vspace*{0.25cm}

\noindent\makebox[\linewidth]{\rule{\textwidth}{0.4pt}}
Estudiante de Ingeniería de Sistemas en la Universidad del Valle con conocimientos en JavaScript, HTML y CSS. Mis habilidades se extienden desde UX, diseño y otras habilidades relacionadas con la experiencia del usuario, incluyendo el dominio de herramientas de diseño como Figma, también puedo trabajar con herramientas colaborativas como Atlassian, Jira y Git. Tengo un nivel de inglés B2. Tengo experiencia trabajando directamente con clientes, realizando levantamiento de requerimientos y usando metodologías ágiles como Scrum, y ceremonias como reuniones diarias, retrospectivas y otras ceremonias ágiles. Estoy buscando una oportunidad para aplicar mis habilidades y conocimientos en un entorno desafiante y dinámico, enfocándome en el desarrollo orientado al cliente para mejorar las experiencias de los usuarios.\\
\noindent\makebox[\linewidth]{\rule{\textwidth}{0.4pt}}

\subsection*{Experiencia}

\vspace*{0.2cm}
\textbf{Diseñadora y Desarrolladora Web Freelance en \href{}{IntelDocs}} (Noviembre 2024 - Presente)
\hfill
\vspace*{0.2cm}
\begin{minipage}{\linewidth}
\begin{itemize}[noitemsep]
    \item Diseñar el sitio web de IntelDocs, una plataforma de gestión de documentos, utilizando Figma, asegurando una interfaz intuitiva y fácil de usar.
    \item Diseñar el modelo de contenido y la estructura del sitio web, asegurando que la información esté organizada y presentada de manera clara y lógica para futuras modificaciones.
    \item Desarrollar el sitio web de IntelDocs, utilizando Astro, asegurando un diseño receptivo y fácil de usar.
    \item Implementar componentes reactivos y animaciones utilizando componentes React embebidos, mejorando la experiencia y el compromiso del usuario.
    \item Colaborar con el equipo de desarrollo para asegurar que el diseño se implemente correctamente y cumpla con los requisitos y expectativas del cliente.
    \item Realizar encuestas de requisitos para recopilar y analizar las necesidades del cliente, asegurando que el producto final se alinee con sus expectativas.
    \item Participar en ceremonias ágiles, como reuniones diarias y retrospectivas, para asegurar la mejora continua y la alineación dentro del equipo.
    \item Crear la página utilizando el framework de renderizado del lado del servidor Astro, asegurando tiempos de carga rápidos, una experiencia de usuario fluida y optimización SEO.
    \item Integrar el sitio web con el CMS \href{https://www.contentful.com/}{contentful}, permitiendo al cliente actualizar y gestionar fácilmente el contenido del sitio web sin necesidad de conocimientos técnicos.
    \item Crear componentes seguros para el renderizado para asegurar que la estructura del sitio web no se vea comprometida por los cambios del cliente, asegurando la integridad y consistencia del sitio web.
    \item Enfocarse en el desarrollo orientado al cliente para mejorar las experiencias de los usuarios optimizando el diseño y la funcionalidad basados en los comentarios del cliente.
\end{itemize}
\hfill
\end{minipage}
Habilidades y herramientas: Figma, HTML, CSS, JavaScript, Astro, React, Contentful, Diseño, Experiencia de Usuario, Interfaz de Usuario, Diseño Responsivo, Optimización SEO, Colaboración, Comunicación, Resolución de Problemas, Creatividad, Gestión del Tiempo, Atención al Detalle, Gestión de Clientes.

\vspace*{0.2cm}
\textbf{Ingeniera de Software en Aux en \href{https://www.univalle.edu.co/}{Universidad del Valle}} (Agosto 2022 - Presente)
\hfill
\vspace*{0.2cm}
\begin{minipage}{\linewidth}
\begin{itemize}[noitemsep]
    \item Desarrollar y mantener informes mensuales automáticos utilizando Google Sheets y Google Apps Script, reduciendo el tiempo dedicado a la elaboración de informes manuales en al menos un 80\%.
    \item Crear plantillas personalizadas para los informes, permitiendo una fácil personalización y adaptación a diferentes departamentos y necesidades.
    \item Mejorar la precisión y fiabilidad de los informes implementando validación de datos y manejo de errores, asegurando la calidad de los datos y los informes.
    \item Liderar el proyecto desde la concepción hasta la implementación, coordinando con las partes interesadas, recopilando requisitos y entregando el producto final a tiempo y dentro del presupuesto.
    \item Interactuar directamente con los clientes para entender sus necesidades y proporcionar actualizaciones regulares sobre el progreso del proyecto.
    \item Optimizar el procesamiento de datos y la lógica del lado del servidor para mejorar las experiencias de los clientes y asegurar una ejecución fluida del producto.
\end{itemize}
\hfill
\end{minipage}
Habilidades y herramientas: Google Sheets, Google Apps Script, Automatización, Análisis de Datos, Visualización de Datos, Gestión de Proyectos, Comunicación, Resolución de Problemas, Recopilación de Requisitos, Diseño Centrado en el Usuario.

\newpage

\vspace*{0.2cm}
\textbf{Desarrolladora Web Freelance en \href{}{SIGEIN S.A.S}} (Junio 2023 - Octubre 2023)
\hfill
\vspace*{0.2cm}
\begin{minipage}{\linewidth}
\begin{itemize}[noitemsep]
    \item Desarrollar interfaces de usuario adaptadas a los requisitos del cliente, asegurando una experiencia intuitiva y fácil de usar.
    \item Colaborar con un equipo enfocado en pruebas rigurosas para minimizar errores y asegurar un producto de alta calidad.
    \item Seguir metodologías ágiles para entregar proyectos a tiempo y dentro del presupuesto, asegurando la satisfacción del cliente.
    \item Participar y colaborar en ceremonias ágiles, como la planificación de sprints, reuniones diarias y retrospectivas, asegurando la alineación y comunicación dentro del equipo.
    \item Realizar encuestas de requisitos para recopilar las necesidades del cliente y asegurar que el producto final cumpla con sus expectativas.
    \item Enfocarse en el desarrollo orientado al cliente para entregar interfaces de usuario adaptadas a los requisitos del cliente, asegurando una experiencia intuitiva y fácil de usar.
\end{itemize}
\hfill
\end{minipage}
Habilidades y herramientas: HTML, CSS, JavaScript, Desarrollo Ágil de Software, Diseño de Interfaces de Usuario, Experiencia de Usuario, Colaboración, Comunicación, Resolución de Problemas, Creatividad, Gestión del Tiempo, Atención al Detalle, Gestión de Clientes, Encuesta de Requisitos, Scrum, Ceremonias Ágiles, Desarrollo Orientado al Cliente.

\noindent\makebox[\linewidth]{\rule{\textwidth}{0.4pt}}

\vspace*{0.2cm}
\textbf{Habilidades Blandas}\\
Autonomía, Trabajo en Equipo, Comunicación, Resolución de Problemas, Creatividad, Gestión del Tiempo, Atención al Detalle, Gestión de Clientes, Liderazgo, Adaptabilidad, Inteligencia Emocional, Pensamiento Crítico, Toma de Decisiones, Resolución de Conflictos, Negociación, Manejo del Estrés, Empatía, Escucha Activa, Persuasión\\

\noindent\makebox[\linewidth]{\rule{\textwidth}{0.4pt}}

\subsection*{Educación}
\vspace*{0.2cm}
\textbf{Ingeniería de Sistemas en la \href{https://www.univalle.edu.co/}{Universidad del Valle}} \hfill

\noindent\makebox[\linewidth]{\rule{\textwidth}{0.4pt}}

\subsection*{Experiencia Académica}
\vspace*{0.2cm}

\begin{tabular*}{\textwidth}{l@{\extracolsep{\fill}}l}
\begin{minipage}{8.5cm}
    \textbf{Inteligencia Artificial}
    \begin{itemize}[noitemsep, topsep=0pt]
    \item \href{https://github.com/Yi0nn/Informe-IA}{Predicción de Cáncer de Mama}
    \item \href{https://github.com/Yi0nn/Proyecto-IA}{Solucionador de Laberintos}
    \end{itemize}
    \hfill
\end{minipage}&
\begin{minipage}{8.5cm}
    \textbf{Complejidad y Optimización}
    \begin{itemize}[noitemsep, topsep=0pt]
    \item \href{https://github.com/Yi0nn/ProyectoADA2}{Optimización de Unidades Térmicas}
    \item \href{https://github.com/MarthoxGJ/ElectricStation}{Estación Eléctrica}
    \end{itemize}
    \hfill
\end{minipage}
\vspace{0.4cm}\\
\begin{minipage}{8.5cm}
    \textbf{DevOps}
    \begin{itemize}[noitemsep, topsep=0pt]
    \item \href{https://github.com/Yi0nn/tallergithubactions}{Taller de GitHub Actions}
    \end{itemize}
    \hfill
\end{minipage} & 
\begin{minipage}{8.5cm}
    \textbf{Desarrollo de API}
    \begin{itemize}[noitemsep, topsep=0pt]
    \item \href{https://github.com/Yi0nn/ejemploapi}{API de Estudiantes}
    \end{itemize}
    \hfill
\end{minipage}
\vspace{0.4cm}\\
\begin{minipage}{8.5cm}
    \textbf{Desarrollo de Software}
    \begin{itemize}[noitemsep, topsep=0pt]
    \item \href{https://github.com/JoseBecerra02/Sweet-Pet-Shop}{Sweet Pet Shop}
    \item \href{https://github.com/RoseRossi/Visual_Novel}{Novela Visual} \href{https://www.figma.com/design/oyYTHggT5Qzmg5J0RZZVwx/Visual_Novel?m=auto&t=cjb55xFfPUILOhhu-1}{(Figma)}
    \end{itemize}
    \hfill
\end{minipage} & 
\begin{minipage}{8.5cm}
    \textbf{Diseño UX}
    \begin{itemize}[noitemsep, topsep=0pt]
    \item \href{https://github.com/CriistiianDM/Red-wheels}{Red Wheels} \href{https://www.figma.com/design/yonGMfbUZSB5YkmaDHPkU5/Desarrollo?m=auto&t=cjb55xFfPUILOhhu-1}{(Figma)}
    \item \href{https://www.figma.com/design/KMyxhoEvgbkZM3mlT51GvK/EcoBuddy?m=auto&t=cjb55xFfPUILOhhu-1}{Eco Buddy}
    \end{itemize}
    \hfill
\end{minipage} \\
\end{tabular*}

\vspace*{0.5cm}
\noindent\makebox[\linewidth]{\rule{\textwidth}{0.4pt}}

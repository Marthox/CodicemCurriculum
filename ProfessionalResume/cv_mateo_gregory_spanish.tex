% ...existing code...
  % Crear una tabla con dos filas, una más pequeña que la otra
  \noindent
  \begin{tabular*}{\textwidth}{l@{\extracolsep{\fill}}l}
    \textbf{\Large \href{https://www.linkedin.com/in/marthox/}{Mateo Gregory}} & \href{mailto:magremenez@gmail.com}{Correo: magremenez@gmail.com}\\
    Ingeniero de Sistemas \& Ciencias de la Computación & \href{https://www.linkedin.com/in/marthox/}{Linkedin: Marthox} \\
    Ingeniero Full-Stack & \href{https://www.marthox.com}{Sitio web: Marthox.com} \\
    Project Manager & Teléfono: (+57) 316 4193085
  \end{tabular*}

  \vspace*{0.25cm}

  \noindent\makebox[\linewidth]{\rule{\textwidth}{0.4pt}}
  Ingeniero de Sistemas de la Universidad del Valle, con dominio del inglés (certificado B2), con conocimientos y experiencia principalmente en Python y JavaScript y algo de experiencia en Java, C\#, Ruby y Elixir. Tengo experiencia en frameworks y bibliotecas como React, NodeJS, Django, Flask, Fast-Api, Phoenix, Svelte y Rails. Control de versiones (Git) y plataformas de alojamiento de control de versiones remotas (GitHub).\\
  \noindent\makebox[\linewidth]{\rule{\textwidth}{0.4pt}}

  \subsection*{Experiencia}

  \vspace*{0.2cm}
  \textbf{Ingeniero de Software en \href{https://www.softserveinc.com/en-us}{SoftServe}} (Febrero 2024 - Actual)
  \hfill
  \vspace*{0.2cm}
  \begin{minipage}{\linewidth}
    \begin{itemize}[noitemsep]
      \item Contribuir al despliegue y mantenimiento de integraciones realizadas para clientes en los campos de biología y medicina, desarrollando nuevas características, corrigiendo errores existentes y refactorizando código heredado.
      \item Gestionar integraciones sin servidor, desarrollando y desplegando nuevas características, corrigiendo errores existentes y refactorizando la base de código, utilizando servicios de AWS como Lambda, API Gateway, S3, DynamoDB y SQS.
      \item Identificar, priorizar y resolver problemas en las integraciones, monitoreando los registros de la aplicación utilizando herramientas como CloudWatch, Datadog, Sumologic y Sentry, logrando un tiempo de actividad del 98\% en las integraciones.
      \item Encontrar y eliminar recursos no utilizados en la nube, reduciendo el costo de la infraestructura y mejorando el rendimiento de la aplicación.
      \item Escribir pruebas de rendimiento para la aplicación, identificando cuellos de botella y mejorando el rendimiento de la aplicación, logrando hasta un 800\% de mejora en el rendimiento en ciertos puntos finales.
      \item Contribuir con charlas y talleres sobre desarrollo de software, servicios en la nube y mejores prácticas en la empresa, logrando una mejor comprensión de las tecnologías utilizadas en la empresa y por nuestros clientes.
      \item Contribuir a proyectos sociales en la empresa, enseñando y guiando a estudiantes en el desarrollo de proyectos de software, logrando una mejor comprensión de las tecnologías utilizadas en la industria y una mejor preparación para el futuro de los estudiantes.
      \item Liderar la comunidad de Python en la empresa, organizando charlas, talleres y hackatones, logrando una mejor comprensión del lenguaje y las mejores prácticas en la empresa.
    \end{itemize}
    \hfill
  \end{minipage}
  Habilidades y herramientas: Python, TypeScript, JavaScript, AWS, Serverless, Lambda, API Gateway, S3, DynamoDB, SQS, CloudWatch, Datadog, Sumologic, Sentry, Git, Desarrollo Ágil, Gestión de Proyectos, Pruebas de Rendimiento.

  \vspace{0.2cm}
  \textbf{Fundador de la Comunidad \href{https://www.parcha.tech/}{ParchaTech}} (Agosto 2024 - Actual)
  \hfill
  \vspace{0.2cm}
  \begin{minipage}{\linewidth}
    \begin{itemize}[noitemsep]
      \item Organizar charlas, talleres y hackatones sobre desarrollo de software, servicios en la nube y mejores prácticas, logrando una mejor comprensión de las tecnologías utilizadas en la industria y una mejor preparación para el futuro de los asistentes.
      \item Crear y mantener una comunidad de desarrolladores de software y amantes de la tecnologia, fomentando la colaboración, el aprendizaje y el crecimiento profesional, logrando una mejor comprensión de las tecnologías utilizadas en la industria y una mejor preparación para el futuro de los miembros.
      \item Colaborar con empresas y universidades para organizar charlas, talleres y hackatones, logrando una mejor comprensión de las tecnologías utilizadas en la industria y una mejor preparación para el futuro de los asistentes.
      \item Colaborar con otras comunidades de desarrollo de software para organizar charlas, talleres y hackatones, fomentando la colaboración, el aprendizaje y el crecimiento profesional, logrando una mejor comprensión de las tecnologías utilizadas en la industria y una mejor preparación para el futuro de los miembros.
    \end{itemize}
    \hfill
  \end{minipage}
  Habilidades y herramientas: Organización de Eventos, Comunicación Efectiva, Colaboración, Liderazgo, Gestión de Comunidades, Desarrollo de Software, Servicios en la Nube, Mejores Prácticas, Tecnologías de la Información.
  \newpage

  \vspace*{0.2cm}
  \textbf{Ingeniero Full-Stack en \href{https://www.inchcape.com/}{Inchcape}} (Octubre 2023 - Febrero 2024)
  \hfill
  \vspace*{0.2cm}
  \begin{minipage}{\linewidth}
    \begin{itemize}[noitemsep]
      \item Contribuir a más de diez sitios web en más de cinco países, manteniendo, desarrollando nuevas características, corrigiendo errores existentes y refactorizando la base de código, logrando abrir nuevos mercados que se espera aumenten los ingresos de la empresa.
      \item Mejorar la experiencia del usuario de los sitios web desarrollando características específicas para cada país y siguiendo principios de diseño centrados en el usuario, aumentando el compromiso del usuario y la tasa de conversión.
    \end{itemize}
    \hfill
  \end{minipage}
  Habilidades y herramientas: C\# (C-Sharp), JavaScript, Html, Css, .Net, Umbraco, Atlassian, Jira, Git, Desarrollo Ágil de Software, Gestión de Proyectos.

  \vspace*{0.2cm}
  \textbf{Profesor de Trabajo de Grado en \href{https://www.univalle.edu.co/}{Universidad del Valle}} (Junio 2023 - Diciembre 2023)
  \hfill
  \vspace*{0.2cm}
  \begin{minipage}{\linewidth}
    \begin{itemize}[noitemsep]
      \item Guiar a los estudiantes en la planificación, desarrollo y presentación de sus trabajos de grado, asegurando la calidad y relevancia de los proyectos.
      \item Proveer retroalimentación constructiva y apoyo técnico en áreas como desarrollo de software, inteligencia artificial y procesamiento de imágenes, entre otras
      \item Evaluar el progreso de los estudiantes y proporcionar orientación para mejorar sus habilidades técnicas y de investigación.
      \item Fomentar un ambiente de aprendizaje colaborativo y motivador, ayudando a los estudiantes a alcanzar sus objetivos académicos y profesionales.
    \end{itemize}
    \hfill
  \end{minipage}
  Habilidades y herramientas: Mentoría, Desarrollo de Software, Inteligencia Artificial, Procesamiento de Imágenes, Investigación Académica, Evaluación de Proyectos, Comunicación Efectiva.

  \vspace*{0.2cm}
  \textbf{Ingeniero Backend en \href{https://reserve.org/}{Reserve}} (Julio 2022 - Mayo 2023)
  \hfill
  \vspace*{0.2cm}
  \begin{minipage}{\linewidth}
    \begin{itemize}[noitemsep]
      \item Optimizar, refactorizar y crear más de diez puntos finales de API, implementando nuevas características y corrigiendo errores existentes, mejorando la estabilidad de la aplicación hasta un 99.98\%.
      \item Abrir la puerta al mercado mexicano integrando un proveedor de remesas de terceros, permitiendo a los usuarios retirar y depositar dinero en efectivo, alcanzando un mercado de \$58.5 mil millones.
      \item Mejorar la robustez de la aplicación en diferentes escenarios de carga de trabajo, implementando una herramienta de pruebas de carga personalizable, permitiendo al equipo identificar cuellos de botella y mejorar el rendimiento de la aplicación.
    \end{itemize}
    \hfill
  \end{minipage}
  Habilidades y herramientas: Python, JavaScript, Html, Css, FastAPI, React, Desarrollo Backend Full-Stack, Desarrollo de API RESTful, Integraciones de API, Git, Desarrollo Ágil de Software, Gestión de Proyectos.

  \vspace*{0.2cm}
  \textbf{Ingeniero Full-Stack en \href{https://kommit.co/}{Kommit}} (Octubre 2021 - Julio 2022)
  \hfill
  \vspace*{0.2cm}
  \begin{minipage}{\linewidth}
    \begin{itemize}[noitemsep]
      \item Mejorar la estabilidad y funcionalidad de \href{https://www.podnation.co/}{Podnation} implementando nuevas características y corrigiendo errores existentes, manteniendo el ya alto tiempo de actividad de la aplicación del 99.99\%.
      \item Lanzar una nueva característica para \href{https://www.podnation.co/}{Podnation} que permite a los usuarios crear y compartir clips de sus podcasts favoritos, aumentando el compromiso del usuario y la tasa de conversión en un 10\%.
      \item Refactorizar y mejorar el controlador de dispositivos de hardware de una Fundación Militar de EE. UU., ayudando significativamente a pacientes con Parkinson con dificultades del habla, logrando una mejora del 99.99\% en la robustez y estabilidad de la aplicación.
    \end{itemize}
    \hfill
  \end{minipage}
  Habilidades y herramientas: JavaScript, Elixir, Ruby, React, AdonisJs, ElectronJs, Phoenix, Ruby on Rails, Desarrollo Full-Stack, Git, Desarrollo Ágil de Software, Gestión de Proyectos.

  \vspace*{0.2cm}
  \textbf{Ingeniero Full-Stack Freelance} (Enero 2021 - Agosto 2021)
  \hfill
  \vspace*{0.2cm}
  \begin{minipage}{\linewidth}
    \begin{itemize}[noitemsep]
      \item Desarrollar una interfaz específica para pacientes, optimizada demográficamente para el público objetivo, implementando una biblioteca de componentes React personalizada, aumentando el compromiso del usuario en un 15\%.
      \item Crear interfaces de usuario intuitivas, optimizadas para el público objetivo demográfico para mejorar la experiencia y el compromiso del usuario, implementando un componente React personalizado, aumentando el compromiso del usuario en un 15\%.
      \item Construir un modelo de base de datos basado en roles para la gestión eficiente de datos de pacientes, médicos y citas, asegurando la integridad y seguridad de los datos.
      \item Supervisar el despliegue, configuración y configuración de la aplicación en servidores de Digital Ocean utilizando Docker, asegurando un proceso de despliegue suave y eficiente.
    \end{itemize}
    \hfill
  \end{minipage}
  Habilidades: JavaScript, Python, SQL, React, Flask, PostgreSQL, Docker, Digital Ocean, Desarrollo Freelance, Desarrollo Full-Stack, Gestión de Proyectos, Git, Desarrollo Ágil de Software, Desarrollo de API RESTful.

  \vspace*{0.2cm}
  \textbf{Habilidades Blandas}\\
  Autonomía, Trabajo en equipo, Orientación a objetivos, Liderazgo, Gestión de conflictos, Habilidades de enseñanza, Escucha activa, Comunicación clara, Resolución de problemas, Flexibilidad, Pensamiento creativo, Gestión del tiempo, Atención al detalle, Enfoque centrado en el usuario, Habilidades analíticas, Inteligencia emocional, Gestión de proyectos, Persistencia.
  
  \subsection*{Educación}
  \vspace*{0.2cm}
  \textbf{Grado en Ingeniería de Sistemas en \href{https://www.univalle.edu.co/}{Universidad del Valle}} \hfill (Junio 2016 - Diciembre 2021)

  \noindent\makebox[\linewidth]{\rule{\textwidth}{0.4pt}}

  \subsection*{Experiencia Académica}
  \vspace*{0.2cm}

  \begin{tabular*}{\textwidth}{l@{\extracolsep{\fill}}l}
    \begin{minipage}{8.5cm}
      \textbf{Complejidad y optimización}
      \begin{itemize}[noitemsep, topsep=0pt]
        \item \href{https://github.com/MarthoxGJ/SATReductor}{SAT Reductor}
        \item \href{https://github.com/MarthoxGJ/ElectricStation}{ElectricS tation}
      \end{itemize}
      \hfill
    \end{minipage} & 
    \begin{minipage}{8.5cm}
      \textbf{Inteligencia Artificial}
      \begin{itemize}[noitemsep, topsep=0pt]
        \item \href{https://github.com/MarthoxGJ/HexAgent}{Hex Agent}
        \item \href{https://github.com/MarthoxGJ/SokobanAgents}{Sokoban Agents}
      \end{itemize}
      \hfill
    \end{minipage}
    \vspace{0.4cm}\\
    \begin{minipage}{8.5cm}
      \textbf{Procesamiento de Imágenes}
      \begin{itemize}[noitemsep, topsep=0pt]
        \item \href{https://github.com/MarthoxGJ/ImageProcessing}{Image processing}
        \item \href{https://github.com/MarthoxGJ/MIA}{MIA}
      \end{itemize}
      \hfill
    \end{minipage} & 
    \begin{minipage}{8.5cm}
      \textbf{Redes}
      \begin{itemize}[noitemsep, topsep=0pt]
        \item \href{https://github.com/MarthoxGJ/GNS3Topology}{GNS3 topology}
      \end{itemize}
      \hfill
    \end{minipage}
    \vspace{0.4cm}\\
    \begin{minipage}{8.5cm}
      \textbf{Desarrollo de Software}
      \begin{itemize}[noitemsep, topsep=0pt]
        \item \href{https://github.com/MarthoxGJ/UVEnergy}{UV energy}
        \item \href{https://github.com/MarthoxGJ/Nova}{Nova}
      \end{itemize}
      \hfill
    \end{minipage} & 
    \begin{minipage}{8.5cm}
      \textbf{Virtualización}
      \begin{itemize}[noitemsep, topsep=0pt]
        \item \href{https://github.com/MarthoxGJ/VBoxRESTAPI}{Vitual Box Rest API}
      \end{itemize}
      \hfill
    \end{minipage} \\
  \end{tabular*}

  \vspace*{0.2cm}
  \noindent\makebox[\linewidth]{\rule{\textwidth}{0.4pt}}

  \subsection*{Extras}
  \vspace*{0.2cm}
  \begin{itemize}[noitemsep, topsep=0pt]
    \item Monitor de acompañamiento académico en la Universidad del Valle
    \item Monitor de acompañamiento personalizado en la Universidad del Valle
    \item Asistente al taller de gamificación “Pioneras-Dev” por Edwin Gamboa
    \item Trabajo en la redacción de un artículo académico con Valeria Rivera, Emily Carvajal y Jose Bernal
    \item Asistente al 13º Congreso Colombiano de Computación
  \end{itemize}

  \vspace*{0.2cm}
  \noindent\makebox[\linewidth]{\rule{\textwidth}{0.4pt}}
